\documentclass{amsart}
\usepackage[foot]{amsaddr} % put addresses on first page

\usepackage{geometry}
\usepackage{booktabs}
\usepackage{graphicx,psfrag,epsf}
\usepackage{enumerate}
\usepackage{enumitem}
\usepackage{amsfonts}
\usepackage{mathtools}
\usepackage{amssymb}
\usepackage{amsmath}
\allowdisplaybreaks
\usepackage{longtable}
\usepackage{bigints}
\usepackage{siunitx}
\usepackage{amsthm}
\usepackage{soul}
\usepackage{color}

\usepackage{hyperref}
\usepackage[capitalise]{cleveref}
\newtheorem{theorem}{Theorem}[section]
\newtheorem{corollary}{Corollary}[theorem]
\newtheorem{lemma}[theorem]{Lemma}
%

\usepackage[numbers]{natbib}
\usepackage{url} % not crucial - just used below for the URL 
\usepackage{doi}

\title{Copula estimation using loss-based Bayesian Additive Regression Trees}
\author{**}
\date{\today}

\begin{document}

\maketitle

\section{Introduction}

\section{Model}

\subsection{Conditional copula}

Let $Y_1$ and $Y_2$ be two continuous random variables and $X$ be a continuous random variable 
that might affect the relationship between $Y_1$ and $Y_2$. 
Then according to Sklar’s theorem there exists a unique copula such that:
\begin{equation}
    H_{X}(y_1,y_2\mid x; \theta, \alpha_1, \alpha_2) = C\{F_{Y_1\mid X}(y_1\mid x;\alpha_1),F_{Y_2\mid X}(y_2\mid x; \alpha_2)\mid x;\theta\}; \quad \forall (y_1,y_2) \in \mathbb{R}^2.
\end{equation}
This gives us the following
\begin{equation}
    h_{X}(y_1,y_2\mid x; \theta, \alpha_1, \alpha_2) = f_{Y_1\mid X}(y_1\mid x;\alpha_1)f_{Y_2\mid X}(y_2\mid x; \alpha_2)c(u_1,u_2\mid x;\theta),
\end{equation}
where
\begin{equation}\label{eq:emp_dist:Y}
    u_k = F_{Y_k\mid X}(y_k\mid x; \alpha_k)
\end{equation}
and $c(u_1,u_2\mid x;\theta)$ is conditional copula density function.

\subsection{Loss-based BART}

Let $T$ denote a tree and $M$ denote the vector of terminal node values $M = \{\mu_1,\mu_2, \cdots, \mu_{n_L}\}$. Let $\theta_i$ denote the copula parameter conditional on $x_i$ such that:
\begin{equation}
	\theta_i = g(x_i, T, M).
\end{equation}
For, $T,M$ a loss-based prior is proposed by \citet{serafini2024lossbasedpriortreetopologies} in the following way:
\begin{align}
	T, M &\sim \pi(T)\pi(M\mid T)\\
	T &\propto \exp\left(\omega n_L(T)-\gamma\Delta(T)\right)\\
	\pi(M\mid T) & = \prod_{j=1}^{n_L}\pi(\mu_j\mid T)
\end{align}

For the choice of prior on $\mu_j\mid T$ a conjugate prior is suggested by \citet{chipman_BART,serafini2024lossbasedpriortreetopologies}. However, finding a conjugate prior for copula parameter may not be possible always. Instead, we use a flat-prior on the support of the copula parameter and employ Metropolis-Hastings algorithm for obtaining posterior estimates.


\iffalse

\subsubsection{Choice for $\mu_j$}
\begin{itemize}
    \item Jeffrey's prior \url{https://www2.stat.duke.edu/~berger/papers/bivariate.pdf}
    \item Simple uniform distribution in $(-1,1)$ 
    \item Transformed beta \url{https://academic.oup.com/jrsssa/article/145/2/237/7105492}
    \item Log-normal on $-\log(1-p^2)$
    \item Inverse-Gamma on $-\log(1-p^2)$
\end{itemize}
Note that for the first three choices, we only work with transformed beta distribution by 
selecting suitable hyper-parameters.

\subsection{Density of $\rho$}

Let $\kappa = -\log(1-p^2)$, then for $\rho \ge 0$
\begin{align}
	\rho & = \sqrt{1-\exp(-\kappa)}\coloneqq g(\kappa)
\end{align}

Now, let $0\le\kappa_1 < \kappa_2<\infty$, then

\begin{align}
	&g(\kappa_2) - g(\kappa_1)\\
	&=\sqrt{1-\exp(-\kappa_2)} - \sqrt{1-\exp(-\kappa_1)}\\
	&= \frac{(\sqrt{1-\exp(-\kappa_2)} - \sqrt{1-\exp(-\kappa_1)})(\sqrt{1-\exp(-\kappa_2)} + \sqrt{1-\exp(-\kappa_1)})}{(\sqrt{1-\exp(-\kappa_2)} + \sqrt{1-\exp(-\kappa_1)})}\\
	&= \frac{(1-\exp(-\kappa_2) - 1+\exp(-\kappa_1))}{(\sqrt{1-\exp(-\kappa_2)} + \sqrt{1-\exp(-\kappa_1)})}\\
	&= \frac{(\exp(-\kappa_1)-\exp(-\kappa_2))}{(\sqrt{1-\exp(-\kappa_2)} + \sqrt{1-\exp(-\kappa_1)})}\\
	&= \frac{(\exp(\kappa_2)-\exp(-\kappa_1))}
	{(\exp(\kappa_2)\exp(-\kappa_1))(\sqrt{1-\exp(-\kappa_2)} + \sqrt{1-\exp(-\kappa_1)})} >0.
\end{align}
Therefore it is monotone and increasing.

Now, for $\rho \ge 0$ the density is given by
\begin{align}
	f_{\rho}(\rho)&=f_{\kappa}(\kappa)\left|\frac{d \kappa}{d \rho}\right|\\
	&=f_{\kappa}\left(-\log(1-p^2)\right)\frac{2\rho}{1-\rho^2}
\end{align}
Therefore for log-normal distribution
\begin{equation}
	f_{\rho}(\rho)
	= \frac{1}{\sqrt{2\pi\sigma^2}(-\log(1-\rho^2))} 
	\exp\left(-\frac{(\log(-\log(1-\rho^2))-\mu)^2}{2\sigma^2}\right)
	\cdot \frac{2\rho}{1-\rho^2}
\end{equation}
Similarly, we can derive the expression $\rho<0$ using the relation
$\rho = -\sqrt{1-\exp(-\kappa)}$. This gives us the following
symmetric density function
\begin{equation}
	f_{\rho}(\rho)
	= \frac{\sqrt{2}|\rho|}{\sqrt{\pi}\sigma(1-\rho^2)(-\log(1-\rho^2))} 
	\exp\left(-\frac{(\log(-\log(1-\rho^2))-\mu)^2}{2\sigma^2}\right)
\end{equation}

\paragraph{Continuity at 0} To show the continuity of the density function at 0 we first need to show that the limit $f_{\rho}(0)$
exists.

Since,

\begin{align}
	\lim_{\rho\to 0}f_{\rho}(\rho)
	&= \lim_{\rho\to 0}\left[\frac{\sqrt{2}|\rho|}{\sqrt{\pi}\sigma(1-\rho^2)(-\log(1-\rho^2))} 
	\exp\left(-\frac{(\log(-\log(1-\rho^2))-\mu)^2}{2\sigma^2}\right)\right]\\
	&= \frac{\sqrt{2}}{\sqrt{\pi}\sigma}\lim_{\rho\to 0}\left[\frac{|\rho|}{(1-\rho^2)} \right]
	\lim_{\rho\to 0}\left[\frac{1}{(-\log(1-\rho^2))} 
	\exp\left(-\frac{(\log(-\log(1-\rho^2))-\mu)^2}{2\sigma^2}\right)\right]
\end{align}
The limit $\lim_{\rho\to 0}\left[\frac{|\rho|}{(1-\rho^2)} \right]$ exists and equal to zero. For the remaining part we use change of variables.

\begin{align}
	&\lim_{\rho\to 0}\left[\frac{1}{(-\log(1-\rho^2))} 
	\exp\left(-\frac{(\log(-\log(1-\rho^2))-\mu)^2}{2\sigma^2}\right)\right]\\
	& = \lim_{\kappa\to 0}\left[\frac{1}{\kappa} 
	\exp\left(-\frac{(\log(\kappa)-\mu)^2}{2\sigma^2}\right)\right]\\
	& = \lim_{\eta\to -\infty}\left[\frac{1}{\exp(\eta)} 
	\exp\left(-\frac{(\eta-\mu)^2}{2\sigma^2}\right)\right]\\
	& = \lim_{\eta\to -\infty}\left[\frac{1}{\exp(\eta)} 
	\exp\left(-\frac{(\eta-\mu)^2}{2\sigma^2}\right)\right]\quad \text{TB: this holds for finite limits need to verify infinite}\\
	& = \lim_{\eta\to -\infty}\left[ 
	\exp\left(-\frac{(\eta-\mu)^2}{2\sigma^2}-\eta\right)\right] = 0
\end{align}
Since the density function is symmetric around 0, proceeding like before we can show that
\begin{equation}
	\lim_{\rho\to 0+}f_{\rho}(\rho) = \lim_{\rho\to 0-}f_{\rho}(\rho) = 0
\end{equation}

Now, for inverse-gamma distribution, the density of $\kappa$ is given by:
\begin{equation}
	f_{\kappa}(\kappa)= \frac{b^{a}}{\Gamma(a)}
	\kappa^{-a-1}\exp(-b/\kappa)
\end{equation}
Therefore, the density of $\rho$ is given by:
\begin{equation}
	f_{\rho}(\rho) = \frac{2b^{a}|\rho|}{\Gamma(a)(1-\rho^2)}
	(-\log(1-p^2))^{-a-1}\exp\left(-\frac{b}{-\log(1-p^2)}\right)
\end{equation}
Like before, we can show that the function is continuous at 0.

\begin{figure}[ht]
	\centering
	\includegraphics[width=0.85\linewidth]{prior_comparison.pdf}
	\caption{Different probability density functions for $\mu_j$ a)black for uniform distribution,
	b)grey for Beta(2,2),
	c)red for Jeffrey's prior,
	d)orange for Beta(0.5,0.5),
	e)green for inverse-gamma (1,1),
	f)dark green for inverse-gamma (2,2),
	g)blue for log-normal (0,1) and
	h)dark blue for log-normal (0, 0.8)}
	\label{fig:sim-prior}
\end{figure}
\fi

\section{Simulation Studies}

\subsection{Data generation}

To generate the true copula parameter, we first consider 4 different test cases to simulate Kendall's tau conditional on $x$. Such that
 
\begin{itemize}
    \item Case 1: True $\tau_x$ has a tree structure with respect to $x$.
    \item Case 2: True $\tau_x$ is monotone with respect to $x$ such that 
    \begin{equation}\label{eq:synth:tau_x:case2}
        \tau_x = 0.3 + 0.2 \sin(3x) + 0.3x^2.
    \end{equation}
    \item Case 3: True $\tau_x$ is convex with respect to $x$ such that 
    \begin{equation}\label{eq:synth:tau_x:case3}
        \tau_x = 0.5 + 0.3 \sin(3x).
    \end{equation}
    \item Case 4: True $\tau_x$ non-convex and non-monotone with respect to $x$ such that 
    \begin{equation}\label{eq:synth:tau_x:case4}
        \tau_x = 0.6 - 0.3 \sin(2x) + 0.2 \sin(4x) + 0.3 x^2.
    \end{equation}
\end{itemize}

We present the plot of true values of Kendall's $\tau$ with respect to $x$ in \cref{fig:true:tau}.

\begin{figure}
    \centering
    \includegraphics[width=0.95\linewidth]{true_tau.pdf}
    \caption{True values of Kendall's $\tau$ with respect to $x$. The top left plot shows case with tree structure; top right plot shows case defined by \cref{eq:synth:tau_x:case2}; bottom left shows case defined by \cref{eq:synth:tau_x:case3}; and bottom right shows case defined by \cref{eq:synth:tau_x:case4}.}
    \label{fig:true:tau}
\end{figure}

Then using these values of conditional Kendall's $\tau$, we generate the copula parameters using the link functions summarised in \cref{tab:cop:link}. Afterwards, we simulate from the copula density functions to obtain our synthetic dataset.

\begin{table}
    \centering
    \begin{tabular}{l|c|c|c}
    \toprule
        Family & Support & Relation with $\tau$ & Range of $\tau$ \\
         \midrule
        Gaussian & $\rho \in (-1,1)$ & $\sin(\tau\pi/2)$ & (-1,1)\\
        Student-t & $\rho \in (-1,1)$ & $\sin(\tau\pi/2)$ & (-1,1) \\
        Clayton & $\theta \in (0,\infty)$ & $2\tau/(1-\tau)$ & $(0,1)$ \\
        Gumbel & $\theta\in [1,\infty)$ & $1/(1-\tau)$  & $[0,1)$ \\
        \bottomrule
    \end{tabular}
    \caption{Copula families used for analyses, along with parameter support and relation with Kendall's $\tau$.}
    \label{tab:cop:link}
\end{table}

\subsection{Results}

\subsubsection{Gaussian Copula} For the Gaussian copula, the copula parameter $\rho$ lies within the open interval $(-1,1)$. So a natural choice for prior on $\mu_j$ is transformed beta as suggested by \citet{gokhale_prior_cor}. 

\begin{equation}
	\text{TBeta}(a, b) = \frac{1}{2^{a+b-1}\mathcal{B}(a,b)}(1+\rho)^{a-1}(1-\rho)^{b-1},
\end{equation}
for $a,b>0$ and $\mathcal{B}$ denotes beta function. Note that, for $a=b=0$, this leads to an improper prior. 

We present the summary of our analyses with Gaussian copula in \cref{tab:gauss:summary}. 

\begin{table}[ht]
	\centering
	\caption{Summary of analyses with Gaussian copula. The columns represents the specific case, the type of prior on $\mu_j\mid T$, the posterior expected number of terminal nodes, the posterior expected depth, the acceptance rate of MH algorithm, RMSE of estimated $\rho$ against true $\rho$, length of credible interval and coverage frequency within the credible interval. The posterior quantities are obtained by running 15000 samples in a single chain, after that we remove 5000 samples and then it is thinned by 10.}
	\label{tab:gauss:summary}
	\scriptsize{
	\begin{tabular}{ll|cccccc}
		\toprule
		& Prior on $\mu_j$ & $\mathbb{E}(n_L\mid U,X)$ & $\mathbb{E}(D\mid U,X)$ & Acc. Rate & RMSE & CI length & CI coverage \\ 
		\midrule
		Case 1 & TBeta(0.5,0.5) & 4.538 & 2.403 & 0.2390 & 0.0096 & 0.1771 & 0.836 \\ 
		(Tree) & TBeta(0,0) & 4.862 & 2.607 & 0.2560 & 0.0089 & 0.1907 & 0.840 \\ 
		& TBeta(2,2) & 4.563 & 2.424 & 0.2420 & 0.0098 & 0.1948 & \textbf{0.908} \\ 
		& TBeta(1,1) & 4.608 & 2.484 & 0.2510 & 0.0092 & 0.1963 & 0.844 \\ 
		\midrule
		Case 2 & TBeta(0.5,0.5) & 3.176 & 1.814 & 0.2290 & 0.0021 & 0.1891 & 0.960 \\ 
		(\cref{eq:synth:tau_x:case2}) & TBeta(0,0) & 2.371 & 1.300 & 0.2130 & 0.0020 & 0.1592 & 0.968 \\ 
		& TBeta(2,2) & 2.522 & 1.390 & 0.2150 & 0.0020 & 0.1823 & \textbf{1.000} \\ 
		& TBeta(1,1) & 2.397 & 1.306 & 0.1900 & 0.0020 & 0.1625 & 0.930 \\ 
		\midrule
		Case 3 & TBeta(0.5,0.5) & 3.495 & 2.079 & 0.2410 & 0.0010 & 0.0743 & 0.760 \\ 
		(\cref{eq:synth:tau_x:case3}) & TBeta(0,0) & 3.863 & 2.165 & 0.2620 & 0.0009 & 0.0790 & \textbf{0.866} \\ 
		& TBeta(2,2) & 3.596 & 2.142 & 0.2460 & 0.0008 & 0.0808 & 0.790 \\ 
		& TBeta(1,1) & 5.374 & 2.966 & 0.3510 & 0.0009 & 0.0863 & 0.776 \\ 
		\midrule
		Case 4 & TBeta(0.5,0.5) & 2.558 & 1.402 & 0.2190 & 0.0005 & 0.1197 & \textbf{1.000} \\ 
		(\cref{eq:synth:tau_x:case4}) & TBeta(0,0) & 2.934 & 1.632 & 0.2520 & 0.0006 & 0.1227 & 0.972 \\ 
		& TBeta(2,2) & 2.398 & 1.308 & 0.2390 & 0.0005 & 0.1160 & 0.998 \\ 
		& TBeta(1,1) & 2.848 & 1.571 & 0.2120 & 0.0006 & 0.1296 & \textbf{1.000} \\ 
		\end{tabular}}
\end{table}


\paragraph{Trace-plots} We present trace-plots for $n_l$, depth and likelihood for case 1 in \cref{fig:case1:gauss:nterm,fig:case1:gauss:depth,fig:case1:gauss:like}. Similarly for case 2 the trace-plots are provided in \cref{fig:case2:gauss:nterm,fig:case2:gauss:depth,fig:case2:gauss:like}; for case 3 the trace-plots are provided in \cref{fig:case3:gauss:nterm,fig:case3:gauss:depth,fig:case3:gauss:like}; and for case 4 the trace-plots are provided in \cref{fig:case4:gauss:nterm,fig:case4:gauss:depth,fig:case4:gauss:like}.

% Tbeta(2,2) seems to be good, acceptance rate is high and likelihood seems to be more consistent


\begin{figure}
	\centering
	\includegraphics[width = 0.75\linewidth]{trace_case1_gauss_nterm.pdf}
	\caption{Trace plot of $n_L$ for case 1 (tree structure) with Gaussian copula. The top left denotes analysis with TBeta(0.5,0.5), top right denotes analysis with TBeta(0,0), bottom left denotes analysis with TBeta(2,2) and bottom right denotes analysis with TBeta(1,1).}
	\label{fig:case1:gauss:nterm}
\end{figure}

\begin{figure}
	\centering
	\includegraphics[width = 0.75\linewidth]{trace_case1_gauss_depth.pdf}
	\caption{Trace plot of depth for case 1 (tree structure) with Gaussian copula. The top left denotes analysis with TBeta(0.5,0.5), top right denotes analysis with TBeta(0,0), bottom left denotes analysis with TBeta(2,2) and bottom right denotes analysis with TBeta(1,1).}
	\label{fig:case1:gauss:depth}
\end{figure}

\begin{figure}
	\centering
	\includegraphics[width = 0.75\linewidth]{trace_case1_gauss_like.pdf}
	\caption{Trace plot of likelihood for case 1 (tree structure) with Gaussian copula. The top left denotes analysis with TBeta(0.5,0.5), top right denotes analysis with TBeta(0,0), bottom left denotes analysis with TBeta(2,2) and bottom right denotes analysis with TBeta(1,1).}
	\label{fig:case1:gauss:like}
\end{figure}

\begin{figure}
	\centering
	\includegraphics[width = 0.75\linewidth]{trace_case2_gauss_nterm.pdf}
	\caption{Trace plot of $n_L$ for case 2 (\cref{eq:synth:tau_x:case2}) with Gaussian copula. The top left denotes analysis with TBeta(0.5,0.5), top right denotes analysis with TBeta(0,0), bottom left denotes analysis with TBeta(2,2) and bottom right denotes analysis with TBeta(1,1).}
	\label{fig:case2:gauss:nterm}
\end{figure}

\begin{figure}
	\centering
	\includegraphics[width = 0.75\linewidth]{trace_case2_gauss_depth.pdf}
	\caption{Trace plot of depth for case 2 (\cref{eq:synth:tau_x:case2}) with Gaussian copula. The top left denotes analysis with TBeta(0.5,0.5), top right denotes analysis with TBeta(0,0), bottom left denotes analysis with TBeta(2,2) and bottom right denotes analysis with TBeta(1,1).}
	\label{fig:case2:gauss:depth}
\end{figure}

\begin{figure}
	\centering
	\includegraphics[width = 0.75\linewidth]{trace_case2_gauss_like.pdf}
	\caption{Trace plot of likelihood for case 2 (\cref{eq:synth:tau_x:case2}) with Gaussian copula. The top left denotes analysis with TBeta(0.5,0.5), top right denotes analysis with TBeta(0,0), bottom left denotes analysis with TBeta(2,2) and bottom right denotes analysis with TBeta(1,1).}
	\label{fig:case2:gauss:like}
\end{figure}

\begin{figure}
	\centering
	\includegraphics[width = 0.75\linewidth]{trace_case3_gauss_nterm.pdf}
	\caption{Trace plot of $n_L$ for case 3 (\cref{eq:synth:tau_x:case3}) with Gaussian copula. The top left denotes analysis with TBeta(0.5,0.5), top right denotes analysis with TBeta(0,0), bottom left denotes analysis with TBeta(2,2) and bottom right denotes analysis with TBeta(1,1).}
	\label{fig:case3:gauss:nterm}
\end{figure}

\begin{figure}
	\centering
	\includegraphics[width = 0.75\linewidth]{trace_case3_gauss_depth.pdf}
	\caption{Trace plot of depth for case 3 (\cref{eq:synth:tau_x:case3}) with Gaussian copula. The top left denotes analysis with TBeta(0.5,0.5), top right denotes analysis with TBeta(0,0), bottom left denotes analysis with TBeta(2,2) and bottom right denotes analysis with TBeta(1,1).}
	\label{fig:case3:gauss:depth}
\end{figure}

\begin{figure}
	\centering
	\includegraphics[width = 0.75\linewidth]{trace_case3_gauss_like.pdf}
	\caption{Trace plot of likelihood for case 3 (\cref{eq:synth:tau_x:case3}) with Gaussian copula. The top left denotes analysis with TBeta(0.5,0.5), top right denotes analysis with TBeta(0,0), bottom left denotes analysis with TBeta(2,2) and bottom right denotes analysis with TBeta(1,1).}
	\label{fig:case3:gauss:like}
\end{figure}

\begin{figure}
	\centering
	\includegraphics[width = 0.75\linewidth]{trace_case4_gauss_nterm.pdf}
	\caption{Trace plot of $n_L$ for case 4 (\cref{eq:synth:tau_x:case4}) with Gaussian copula. The top left denotes analysis with TBeta(0.5,0.5), top right denotes analysis with TBeta(0,0), bottom left denotes analysis with TBeta(2,2) and bottom right denotes analysis with TBeta(1,1).}
	\label{fig:case4:gauss:nterm}
\end{figure}

\begin{figure}
	\centering
	\includegraphics[width = 0.75\linewidth]{trace_case4_gauss_depth.pdf}
	\caption{Trace plot of depth for case 4 (\cref{eq:synth:tau_x:case4}) with Gaussian copula. The top left denotes analysis with TBeta(0.5,0.5), top right denotes analysis with TBeta(0,0), bottom left denotes analysis with TBeta(2,2) and bottom right denotes analysis with TBeta(1,1).}
	\label{fig:case4:gauss:depth}
\end{figure}

\begin{figure}
	\centering
	\includegraphics[width = 0.75\linewidth]{trace_case4_gauss_like.pdf}
	\caption{Trace plot of likelihood for case 4 (\cref{eq:synth:tau_x:case4}) with Gaussian copula. The top left denotes analysis with TBeta(0.5,0.5), top right denotes analysis with TBeta(0,0), bottom left denotes analysis with TBeta(2,2) and bottom right denotes analysis with TBeta(1,1).}
	\label{fig:case4:gauss:like}
\end{figure}

\subsubsection{Student-t copula} For the student-t copula we are only interested in estimating $\rho$ which lies within $(-1,1)$. Therefore, similar to Gaussian copula, we use a transformed beta distribution as a prior on $\mu_j$ with four different sets of parameters.

We present the summary of our analyses with student-t copula in \cref{tab:student-t:summary}. 


\begin{table}[ht]
	\centering
	\caption{Summary of analyses with student-t copula. The columns represents the specific case, the type of prior on $\mu_j\mid T$, the posterior expected number of terminal nodes, the posterior expected depth, the acceptance rate of MH algorithm, RMSE of estimated $\rho$ against true $\rho$, length of credible interval and coverage frequency within the credible interval. The posterior quantities are obtained by running 15000 samples in a single chain, after that we remove 5000 samples and then it is thinned by 10.}
	\label{tab:student-t:summary}
	\scriptsize{
	\begin{tabular}{ll|cccccc}
		\toprule
		& Prior on $\mu_j$ & $\mathbb{E}(n_L\mid U,X)$ & $\mathbb{E}(D\mid U,X)$ & Acc. Rate & RMSE & CI length & CI coverage \\ 
		\midrule
		Case 1 & TBeta(0.5,0.5) & 3.370 & 1.884 & 0.251 & 0.0079 & 0.2277 & 0.860 \\ 
		(Tree) & TBeta(0,0) & 4.089 & 2.228 & 0.268 & 0.0072 & 0.2445 & 0.850 \\ 
		& TBeta(2,2) & 3.620 & 2.041 & 0.260 & 0.0080 & 0.2420 & \textbf{0.862} \\ 
		& TBeta(1,1) & 3.604 & 1.996 & 0.232 & 0.0080 & 0.2308 & 0.854 \\ 
		\midrule
		Case 2 & TBeta(0.5,0.5) & 2.394 & 1.311 & 0.246 & 0.0019 & 0.1980 & \textbf{1.000} \\ 
		(\cref{eq:synth:tau_x:case2}) & TBeta(0,0) & 2.326 & 1.255 & 0.203 & 0.0018 & 0.2000 & \textbf{1.000} \\ 
		& TBeta(2,2) & 2.470 & 1.351 & 0.236 & 0.0019 & 0.2115 & \textbf{1.000} \\ 
		& TBeta(1,1) & 2.354 & 1.289 & 0.213 & 0.0018 & 0.1950 & \textbf{1.000} \\ 
		\midrule
		Case 3 & TBeta(0.5,0.5) & 3.499 & 2.102 & 0.261 & 0.0008 & 0.0997 & 0.916 \\ 
		(\cref{eq:synth:tau_x:case3}) & TBeta(0,0) & 3.727 & 2.170 & 0.261 & 0.0007 & 0.0992 & 0.928 \\ 
		& TBeta(2,2) & 3.545 & 2.127 & 0.238 & 0.0007 & 0.1127 & 0.896 \\ 
		& TBeta(1,1) & 3.847 & 2.234 & 0.243 & 0.0008 & 0.0971 & \textbf{0.950} \\ 
		\midrule
		Case 4 & TBeta(0.5,0.5) & 2.560 & 1.431 & 0.241 & 0.0003 & 0.1552 & \textbf{1.000} \\ 
		(\cref{eq:synth:tau_x:case4}) & TBeta(0,0) & 2.387 & 1.311 & 0.275 & 0.0003 & 0.1406 & \textbf{1.000} \\ 
		& TBeta(2,2) & 2.689 & 1.488 & 0.266 & 0.0003 & 0.1648 & \textbf{1.000} \\ 
		& TBeta(1,1) & 2.824 & 1.568 & 0.259 & 0.0003 & 0.1482 & \textbf{1.000} \\ 
		\end{tabular}}
\end{table}

\paragraph{Trace-plots} We present trace-plots for $n_l$, depth and likelihood for case 1 in \cref{fig:case1:t:nterm,fig:case1:t:depth,fig:case1:t:like}. Similarly for case 2 the trace-plots are provided in \cref{fig:case2:t:nterm,fig:case2:t:depth,fig:case2:t:like}; for case 3 the trace-plots are provided in \cref{fig:case3:t:nterm,fig:case3:t:depth,fig:case3:t:like}; and for case 4 the trace-plots are provided in \cref{fig:case4:t:nterm,fig:case4:t:depth,fig:case4:t:like}.

% Tbeta(0,0) seems to be good, acceptance rate is high and likelihood seems to be more consistent

\begin{figure}
	\centering
	\includegraphics[width = 0.75\linewidth]{trace_case1_t_nterm.pdf}
	\caption{Trace plot of $n_L$ for case 1 (tree structure) with t copula. The top left denotes analysis with TBeta(0.5,0.5), top right denotes analysis with TBeta(0,0), bottom left denotes analysis with TBeta(2,2) and bottom right denotes analysis with TBeta(1,1).}
	\label{fig:case1:t:nterm}
\end{figure}

\begin{figure}
	\centering
	\includegraphics[width = 0.75\linewidth]{trace_case1_t_depth.pdf}
	\caption{Trace plot of depth for case 1 (tree structure) with t copula. The top left denotes analysis with TBeta(0.5,0.5), top right denotes analysis with TBeta(0,0), bottom left denotes analysis with TBeta(2,2) and bottom right denotes analysis with TBeta(1,1).}
	\label{fig:case1:t:depth}
\end{figure}

\begin{figure}
	\centering
	\includegraphics[width = 0.75\linewidth]{trace_case1_t_like.pdf}
	\caption{Trace plot of likelihood for case 1 (tree structure) with t copula. The top left denotes analysis with TBeta(0.5,0.5), top right denotes analysis with TBeta(0,0), bottom left denotes analysis with TBeta(2,2) and bottom right denotes analysis with TBeta(1,1).}
	\label{fig:case1:t:like}
\end{figure}

\begin{figure}
	\centering
	\includegraphics[width = 0.75\linewidth]{trace_case2_t_nterm.pdf}
	\caption{Trace plot of $n_L$ for case 2 (\cref{eq:synth:tau_x:case2}) with t copula. The top left denotes analysis with TBeta(0.5,0.5), top right denotes analysis with TBeta(0,0), bottom left denotes analysis with TBeta(2,2) and bottom right denotes analysis with TBeta(1,1).}
	\label{fig:case2:t:nterm}
\end{figure}

\begin{figure}
	\centering
	\includegraphics[width = 0.75\linewidth]{trace_case2_t_depth.pdf}
	\caption{Trace plot of depth for case 2 (\cref{eq:synth:tau_x:case2}) with t copula. The top left denotes analysis with TBeta(0.5,0.5), top right denotes analysis with TBeta(0,0), bottom left denotes analysis with TBeta(2,2) and bottom right denotes analysis with TBeta(1,1).}
	\label{fig:case2:t:depth}
\end{figure}

\begin{figure}
	\centering
	\includegraphics[width = 0.75\linewidth]{trace_case2_t_like.pdf}
	\caption{Trace plot of likelihood for case 2 (\cref{eq:synth:tau_x:case2}) with t copula. The top left denotes analysis with TBeta(0.5,0.5), top right denotes analysis with TBeta(0,0), bottom left denotes analysis with TBeta(2,2) and bottom right denotes analysis with TBeta(1,1).}
	\label{fig:case2:t:like}
\end{figure}

\begin{figure}
	\centering
	\includegraphics[width = 0.75\linewidth]{trace_case3_t_nterm.pdf}
	\caption{Trace plot of $n_L$ for case 3 (\cref{eq:synth:tau_x:case3}) with t copula. The top left denotes analysis with TBeta(0.5,0.5), top right denotes analysis with TBeta(0,0), bottom left denotes analysis with TBeta(2,2) and bottom right denotes analysis with TBeta(1,1).}
	\label{fig:case3:t:nterm}
\end{figure}

\begin{figure}
	\centering
	\includegraphics[width = 0.75\linewidth]{trace_case3_t_depth.pdf}
	\caption{Trace plot of depth for case 3 (\cref{eq:synth:tau_x:case3}) with t copula. The top left denotes analysis with TBeta(0.5,0.5), top right denotes analysis with TBeta(0,0), bottom left denotes analysis with TBeta(2,2) and bottom right denotes analysis with TBeta(1,1).}
	\label{fig:case3:t:depth}
\end{figure}

\begin{figure}
	\centering
	\includegraphics[width = 0.75\linewidth]{trace_case3_t_like.pdf}
	\caption{Trace plot of likelihood for case 3 (\cref{eq:synth:tau_x:case3}) with t copula. The top left denotes analysis with TBeta(0.5,0.5), top right denotes analysis with TBeta(0,0), bottom left denotes analysis with TBeta(2,2) and bottom right denotes analysis with TBeta(1,1).}
	\label{fig:case3:t:like}
\end{figure}

\begin{figure}
	\centering
	\includegraphics[width = 0.75\linewidth]{trace_case4_t_nterm.pdf}
	\caption{Trace plot of $n_L$ for case 4 (\cref{eq:synth:tau_x:case4}) with t copula. The top left denotes analysis with TBeta(0.5,0.5), top right denotes analysis with TBeta(0,0), bottom left denotes analysis with TBeta(2,2) and bottom right denotes analysis with TBeta(1,1).}
	\label{fig:case4:t:nterm}
\end{figure}

\begin{figure}
	\centering
	\includegraphics[width = 0.75\linewidth]{trace_case4_t_depth.pdf}
	\caption{Trace plot of depth for case 4 (\cref{eq:synth:tau_x:case4}) with t copula. The top left denotes analysis with TBeta(0.5,0.5), top right denotes analysis with TBeta(0,0), bottom left denotes analysis with TBeta(2,2) and bottom right denotes analysis with TBeta(1,1).}
	\label{fig:case4:t:depth}
\end{figure}

\begin{figure}
	\centering
	\includegraphics[width = 0.75\linewidth]{trace_case4_t_like.pdf}
	\caption{Trace plot of likelihood for case 4 (\cref{eq:synth:tau_x:case4}) with t copula. The top left denotes analysis with TBeta(0.5,0.5), top right denotes analysis with TBeta(0,0), bottom left denotes analysis with TBeta(2,2) and bottom right denotes analysis with TBeta(1,1).}
	\label{fig:case4:t:like}
\end{figure}

\subsubsection{Clayton copula} The copula parameter of the Gumbel copula lies in the open interval $(0,\infty)$. So we consider log-normal and inverse-gamma distribution for $\mu_j$

We present the summary of our analyses with Clayton copula in \cref{tab:clayton:summary}. 


\begin{table}[ht]
	\centering
	\caption{Summary of analyses with Clayton copula. The columns represents the specific case, the type of prior on $\mu_j\mid T$, the posterior expected number of terminal nodes, the posterior expected depth, the acceptance rate of MH algorithm, RMSE of estimated $\theta$ against true $\theta$, length of credible interval and coverage frequency within the credible interval. The posterior quantities are obtained by running 15000 samples in a single chain, after that we remove 5000 samples and then it is thinned by 10.}
	\label{tab:clayton:summary}
	\scriptsize{
		\begin{tabular}{ll|cccccc}
			\toprule
			& Prior on $\mu_j$ & $\mathbb{E}(n_L\mid U,X)$ & $\mathbb{E}(D\mid U,X)$ & Acc. Rate & RMSE & CI length & CI coverage \\ 
			\midrule
			Case 1 & Gamma(1,1) & 2.867 & 1.662 & 0.232 & 0.5097 & 1.3417 & 0.764 \\ 
			(Tree) & Gamma(2,2) & 4.246 & 2.416 & 0.300 & 0.3705 & 1.5866 & \textbf{0.924} \\ 
			& LogNorm(0,1) & 3.629 & 2.095 & 0.285 & 0.4099 & 1.5664 & 0.824 \\ 
			& LogNorm(0,5) & 3.876 & 2.178 & 0.246 & 0.3598 & 1.7381 & 0.842 \\ 
			\midrule
			Case 2 & Gamma(1,1) & 2.291 & 1.228 & 0.202 & 0.1254 & 0.9802 & 0.880 \\ 
			(\cref{eq:synth:tau_x:case2}) & Gamma(2,2) & 2.703 & 1.513 & 0.245 & 0.0902 & 1.2297 & \textbf{1.000} \\ 
			& LogNorm(0,1) & 2.799 & 1.581 & 0.249 & 0.0757 & 1.2844 & \textbf{1.000} \\ 
			& LogNorm(0,5) & 2.625 & 1.467 & 0.213 & 0.0834 & 1.2934 & \textbf{1.000} \\ 
			\midrule
			Case 3 & Gamma(1,1) & 3.388 & 2.060 & 0.229 & 0.6166 & 2.2269 & 0.754 \\ 
			(\cref{eq:synth:tau_x:case3}) & Gamma(2,2) & 3.588 & 2.125 & 0.259 & 0.5407 & 2.5436 & 0.822 \\ 
			& LogNorm(0,1) & 3.679 & 2.147 & 0.232 & 0.5492 & 2.4543 & \textbf{0.902} \\ 
			& LogNorm(0,5) & 3.516 & 2.111 & 0.265 & 0.5881 & 2.4645 & 0.884 \\ 
			\midrule
			Case 4 & Gamma(1,1) & 2.205 & 1.183 & 0.207 & 0.0492 & 1.0143 & \textbf{0.978} \\ 
			(\cref{eq:synth:tau_x:case4}) & Gamma(2,2) & 2.318 & 1.260 & 0.214 & 0.0527 & 1.0409 & 0.974 \\ 
			& LogNorm(0,1) & 2.377 & 1.301 & 0.241 & 0.0531 & 1.0920 & 0.972 \\ 
			& LogNorm(0,5) & 2.407 & 1.320 & 0.204 & 0.0561 & 1.1192 & 0.974 \\ 
	\end{tabular}}
\end{table}


\paragraph{Trace-plots} We present trace-plots for $n_l$, depth and likelihood for case 1 in \cref{fig:case1:clayton:nterm,fig:case1:clayton:depth,fig:case1:clayton:like}. Similarly for case 2 the trace-plots are provided in \cref{fig:case2:clayton:nterm,fig:case2:clayton:depth,fig:case2:clayton:like}; for case 3 the trace-plots are provided in \cref{fig:case3:clayton:nterm,fig:case3:clayton:depth,fig:case3:clayton:like}; and for case 4 the trace-plots are provided in \cref{fig:case4:clayton:nterm,fig:case4:clayton:depth,fig:case4:clayton:like}.

% LogNorm(0,1) seems nice, acceptance rate is over-all better than others and likelihood seems more consistent

\begin{figure}
	\centering
	\includegraphics[width = 0.75\linewidth]{trace_case1_clayton_nterm.pdf}
	\caption{Trace plot of $n_L$ for case 1 (tree structure) with Clayton copula. The top left denotes analysis with InvGamma(1,1), top right denotes analysis with InvGamma(2,2), bottom left denotes analysis with LogNormal(0,1) and bottom right denotes analysis with LogNormal(0,5).}
	\label{fig:case1:clayton:nterm}
\end{figure}

\begin{figure}
	\centering
	\includegraphics[width = 0.75\linewidth]{trace_case1_clayton_depth.pdf}
	\caption{Trace plot of depth for case 1 (tree structure) with Clayton copula. The top left denotes analysis with InvGamma(1,1), top right denotes analysis with InvGamma(2,2), bottom left denotes analysis with LogNormal(0,1) and bottom right denotes analysis with LogNormal(0,5).}
	\label{fig:case1:clayton:depth}
\end{figure}

\begin{figure}
	\centering
	\includegraphics[width = 0.75\linewidth]{trace_case1_clayton_like.pdf}
	\caption{Trace plot of likelihood for case 1 (tree structure) with Clayton copula. The top left denotes analysis with InvGamma(1,1), top right denotes analysis with InvGamma(2,2), bottom left denotes analysis with LogNormal(0,1) and bottom right denotes analysis with LogNormal(0,5).}
	\label{fig:case1:clayton:like}
\end{figure}

\begin{figure}
	\centering
	\includegraphics[width = 0.75\linewidth]{trace_case2_clayton_nterm.pdf}
	\caption{Trace plot of $n_L$ for case 2 (\cref{eq:synth:tau_x:case2}) with Clayton copula. The top left denotes analysis with InvGamma(1,1), top right denotes analysis with InvGamma(2,2), bottom left denotes analysis with LogNormal(0,1) and bottom right denotes analysis with LogNormal(0,5).}
	\label{fig:case2:clayton:nterm}
\end{figure}

\begin{figure}
	\centering
	\includegraphics[width = 0.75\linewidth]{trace_case2_clayton_depth.pdf}
	\caption{Trace plot of depth for case 2 (\cref{eq:synth:tau_x:case2}) with Clayton copula. The top left denotes analysis with InvGamma(1,1), top right denotes analysis with InvGamma(2,2), bottom left denotes analysis with LogNormal(0,1) and bottom right denotes analysis with LogNormal(0,5).}
	\label{fig:case2:clayton:depth}
\end{figure}

\begin{figure}
	\centering
	\includegraphics[width = 0.75\linewidth]{trace_case2_clayton_like.pdf}
	\caption{Trace plot of likelihood for case 2 (\cref{eq:synth:tau_x:case2}) with Clayton copula. The top left denotes analysis with InvGamma(1,1), top right denotes analysis with InvGamma(2,2), bottom left denotes analysis with LogNormal(0,1) and bottom right denotes analysis with LogNormal(0,5).}
	\label{fig:case2:clayton:like}
\end{figure}

\begin{figure}
	\centering
	\includegraphics[width = 0.75\linewidth]{trace_case3_clayton_nterm.pdf}
	\caption{Trace plot of $n_L$ for case 3 (\cref{eq:synth:tau_x:case3}) with Clayton copula. The top left denotes analysis with InvGamma(1,1), top right denotes analysis with InvGamma(2,2), bottom left denotes analysis with LogNormal(0,1) and bottom right denotes analysis with LogNormal(0,5).}
	\label{fig:case3:clayton:nterm}
\end{figure}

\begin{figure}
	\centering
	\includegraphics[width = 0.75\linewidth]{trace_case3_clayton_depth.pdf}
	\caption{Trace plot of depth for case 3 (\cref{eq:synth:tau_x:case3}) with Clayton copula. The top left denotes analysis with InvGamma(1,1), top right denotes analysis with InvGamma(2,2), bottom left denotes analysis with LogNormal(0,1) and bottom right denotes analysis with LogNormal(0,5).}
	\label{fig:case3:clayton:depth}
\end{figure}

\begin{figure}
	\centering
	\includegraphics[width = 0.75\linewidth]{trace_case3_clayton_like.pdf}
	\caption{Trace plot of likelihood for case 3 (\cref{eq:synth:tau_x:case3}) with Clayton copula. The top left denotes analysis with InvGamma(1,1), top right denotes analysis with InvGamma(2,2), bottom left denotes analysis with LogNormal(0,1) and bottom right denotes analysis with LogNormal(0,5).}
	\label{fig:case3:clayton:like}
\end{figure}

\begin{figure}
	\centering
	\includegraphics[width = 0.75\linewidth]{trace_case4_clayton_nterm.pdf}
	\caption{Trace plot of $n_L$ for case 4 (\cref{eq:synth:tau_x:case4}) with Clayton copula. The top left denotes analysis with InvGamma(1,1), top right denotes analysis with InvGamma(2,2), bottom left denotes analysis with LogNormal(0,1) and bottom right denotes analysis with LogNormal(0,5).}
	\label{fig:case4:clayton:nterm}
\end{figure}

\begin{figure}
	\centering
	\includegraphics[width = 0.75\linewidth]{trace_case4_clayton_depth.pdf}
	\caption{Trace plot of depth for case 4 (\cref{eq:synth:tau_x:case4}) with Clayton copula. The top left denotes analysis with InvGamma(1,1), top right denotes analysis with InvGamma(2,2), bottom left denotes analysis with LogNormal(0,1) and bottom right denotes analysis with LogNormal(0,5).}
	\label{fig:case4:clayton:depth}
\end{figure}

\begin{figure}
	\centering
	\includegraphics[width = 0.75\linewidth]{trace_case4_clayton_like.pdf}
	\caption{Trace plot of likelihood for case 4 (\cref{eq:synth:tau_x:case4}) with Clayton copula. The top left denotes analysis with InvGamma(1,1), top right denotes analysis with InvGamma(2,2), bottom left denotes analysis with LogNormal(0,1) and bottom right denotes analysis with LogNormal(0,5).}
	\label{fig:case4:clayton:like}
\end{figure}

\subsubsection{Gumbel copula} The copula parameter of the Gumbel copula lies in the open interval $[1,\infty)$. So we consider log-normal and inverse-gamma distribution for $\mu_j$.

We present the summary of our analyses with Gumbel copula in \cref{tab:gumbel:summary}. 


\begin{table}[ht]
	\centering
	\caption{Summary of analyses with Gumbel copula. The columns represents the specific case, the type of prior on $\mu_j\mid T$, the posterior expected number of terminal nodes, the posterior expected depth, the acceptance rate of MH algorithm, RMSE of estimated $\theta$ against true $\theta$, length of credible interval and coverage frequency within the credible interval. The posterior quantities are obtained by running 15000 samples in a single chain, after that we remove 5000 samples and then it is thinned by 10.}
	\label{tab:gumbel:summary}
	\scriptsize{
	\begin{tabular}{ll|cccccc}
		\toprule
		& Prior on $\mu_j$ & $\mathbb{E}(n_L\mid U,X)$ & $\mathbb{E}(D\mid U,X)$ & Acc. Rate & RMSE & CI length & CI coverage \\ 
		\hline
		Case 1 & Gamma(1,1) & 4.513 & 2.397 & 0.246 & 0.1292 & 0.9088 & 0.908 \\ 
		(Tree) & Gamma(2,2) & 4.714 & 2.532 & 0.256 & 0.1407 & 0.8179 & 0.858 \\ 
		& LogNorm(0,1) & 4.626 & 2.525 & 0.260 & 0.1321 & 0.9956 & \textbf{0.916} \\ 
		& LogNorm(0,5) & 4.389 & 2.335 & 0.227 & 0.1353 & 0.8653 & 0.856 \\ 
		\midrule
		Case 2 & Gamma(1,1) & 2.373 & 1.289 & 0.202 & 0.0277 & 0.6032 & 0.998 \\ 
		(\cref{eq:synth:tau_x:case2}) & Gamma(2,2) & 2.605 & 1.455 & 0.209 & 0.0262 & 0.6956 & \textbf{1.000} \\ 
		& LogNorm(0,1) & 2.349 & 1.268 & 0.214 & 0.0273 & 0.6049 & 0.994 \\ 
		& LogNorm(0,5) & 2.408 & 1.314 & 0.212 & 0.0277 & 0.6025 & 0.974 \\ 
		\midrule
		Case 3 & Gamma(1,1) & 3.421 & 2.086 & 0.241 & 0.1523 & 1.1744 & 0.764 \\ 
		(\cref{eq:synth:tau_x:case3}) & Gamma(2,2) & 3.429 & 2.077 & 0.252 & 0.1509 & 1.2093 & 0.788 \\ 
		& LogNorm(0,1) & 3.631 & 2.173 & 0.267 & 0.1566 & 1.1954 & 0.792 \\ 
		& LogNorm(0,5) & 3.590 & 2.119 & 0.262 & 0.1575 & 1.2772 & \textbf{0.854} \\ 
		\midrule
		Case 4 & Gamma(1,1) & 2.706 & 1.517 & 0.247 & 0.0152 & 0.7585 & 0.974 \\ 
		(\cref{eq:synth:tau_x:case4}) & Gamma(2,2) & 2.602 & 1.466 & 0.252 & 0.0140 & 0.7451 & 0.974 \\ 
		& LogNorm(0,1) & 2.527 & 1.405 & 0.224 & 0.0161 & 0.7377 & 0.988 \\ 
		& LogNorm(0,5) & 2.445 & 1.353 & 0.217 & 0.0148 & 0.7090 & \textbf{0.990} \\ 
		\end{tabular}}
\end{table}

\paragraph{Trace-plots} We present trace-plots for $n_l$, depth and likelihood for case 1 in \cref{fig:case1:gumbel:nterm,fig:case1:gumbel:depth,fig:case1:gumbel:like}. Similarly for case 2 the trace-plots are provided in \cref{fig:case2:gumbel:nterm,fig:case2:gumbel:depth,fig:case2:gumbel:like}; for case 3 the trace-plots are provided in \cref{fig:case3:gumbel:nterm,fig:case3:gumbel:depth,fig:case3:gumbel:like}; and for case 4 the trace-plots are provided in \cref{fig:case4:gumbel:nterm,fig:case4:gumbel:depth,fig:case4:gumbel:like}.

% LogNorm(0,1) seems nice, acceptance rate is over-all better than others and likelihood seems more consistent

\begin{figure}
	\centering
	\includegraphics[width = 0.75\linewidth]{trace_case1_gumbel_nterm.pdf}
	\caption{Trace plot of $n_L$ for case 1 (tree structure) with Gumbel copula. The top left denotes analysis with InvGamma(1,1), top right denotes analysis with InvGamma(2,2), bottom left denotes analysis with LogNormal(0,1) and bottom right denotes analysis with LogNormal(0,5).}
	\label{fig:case1:gumbel:nterm}
\end{figure}

\begin{figure}
	\centering
	\includegraphics[width = 0.75\linewidth]{trace_case1_gumbel_depth.pdf}
	\caption{Trace plot of depth for case 1 (tree structure) with Gumbel copula. The top left denotes analysis with InvGamma(1,1), top right denotes analysis with InvGamma(2,2), bottom left denotes analysis with LogNormal(0,1) and bottom right denotes analysis with LogNormal(0,5).}
	\label{fig:case1:gumbel:depth}
\end{figure}

\begin{figure}
	\centering
	\includegraphics[width = 0.75\linewidth]{trace_case1_gumbel_like.pdf}
	\caption{Trace plot of likelihood for case 1 (tree structure) with Gumbel copula. The top left denotes analysis with InvGamma(1,1), top right denotes analysis with InvGamma(2,2), bottom left denotes analysis with LogNormal(0,1) and bottom right denotes analysis with LogNormal(0,5).}
	\label{fig:case1:gumbel:like}
\end{figure}

\begin{figure}
	\centering
	\includegraphics[width = 0.75\linewidth]{trace_case2_gumbel_nterm.pdf}
	\caption{Trace plot of $n_L$ for case 2 (\cref{eq:synth:tau_x:case2}) with Gumbel copula. The top left denotes analysis with InvGamma(1,1), top right denotes analysis with InvGamma(2,2), bottom left denotes analysis with LogNormal(0,1) and bottom right denotes analysis with LogNormal(0,5).}
	\label{fig:case2:gumbel:nterm}
\end{figure}

\begin{figure}
	\centering
	\includegraphics[width = 0.75\linewidth]{trace_case2_gumbel_depth.pdf}
	\caption{Trace plot of depth for case 2 (\cref{eq:synth:tau_x:case2}) with Gumbel copula. The top left denotes analysis with InvGamma(1,1), top right denotes analysis with InvGamma(2,2), bottom left denotes analysis with LogNormal(0,1) and bottom right denotes analysis with LogNormal(0,5).}
	\label{fig:case2:gumbel:depth}
\end{figure}

\begin{figure}
	\centering
	\includegraphics[width = 0.75\linewidth]{trace_case2_gumbel_like.pdf}
	\caption{Trace plot of likelihood for case 2 (\cref{eq:synth:tau_x:case2}) with Gumbel copula. The top left denotes analysis with InvGamma(1,1), top right denotes analysis with InvGamma(2,2), bottom left denotes analysis with LogNormal(0,1) and bottom right denotes analysis with LogNormal(0,5).}
	\label{fig:case2:gumbel:like}
\end{figure}

\begin{figure}
	\centering
	\includegraphics[width = 0.75\linewidth]{trace_case3_gumbel_nterm.pdf}
	\caption{Trace plot of $n_L$ for case 3 (\cref{eq:synth:tau_x:case3}) with Gumbel copula. The top left denotes analysis with InvGamma(1,1), top right denotes analysis with InvGamma(2,2), bottom left denotes analysis with LogNormal(0,1) and bottom right denotes analysis with LogNormal(0,5).}
	\label{fig:case3:gumbel:nterm}
\end{figure}

\begin{figure}
	\centering
	\includegraphics[width = 0.75\linewidth]{trace_case3_gumbel_depth.pdf}
	\caption{Trace plot of depth for case 3 (\cref{eq:synth:tau_x:case3}) with Gumbel copula. The top left denotes analysis with InvGamma(1,1), top right denotes analysis with InvGamma(2,2), bottom left denotes analysis with LogNormal(0,1) and bottom right denotes analysis with LogNormal(0,5).}
	\label{fig:case3:gumbel:depth}
\end{figure}

\begin{figure}
	\centering
	\includegraphics[width = 0.75\linewidth]{trace_case3_gumbel_like.pdf}
	\caption{Trace plot of likelihood for case 3 (\cref{eq:synth:tau_x:case3}) with Gumbel copula. The top left denotes analysis with InvGamma(1,1), top right denotes analysis with InvGamma(2,2), bottom left denotes analysis with LogNormal(0,1) and bottom right denotes analysis with LogNormal(0,5).}
	\label{fig:case3:gumbel:like}
\end{figure}

\begin{figure}
	\centering
	\includegraphics[width = 0.75\linewidth]{trace_case4_gumbel_nterm.pdf}
	\caption{Trace plot of $n_L$ for case 4 (\cref{eq:synth:tau_x:case4}) with Gumbel copula. The top left denotes analysis with InvGamma(1,1), top right denotes analysis with InvGamma(2,2), bottom left denotes analysis with LogNormal(0,1) and bottom right denotes analysis with LogNormal(0,5).}
	\label{fig:case4:gumbel:nterm}
\end{figure}

\begin{figure}
	\centering
	\includegraphics[width = 0.75\linewidth]{trace_case4_gumbel_depth.pdf}
	\caption{Trace plot of depth for case 4 (\cref{eq:synth:tau_x:case4}) with Gumbel copula. The top left denotes analysis with InvGamma(1,1), top right denotes analysis with InvGamma(2,2), bottom left denotes analysis with LogNormal(0,1) and bottom right denotes analysis with LogNormal(0,5).}
	\label{fig:case4:gumbel:depth}
\end{figure}

\begin{figure}
	\centering
	\includegraphics[width = 0.75\linewidth]{trace_case4_gumbel_like.pdf}
	\caption{Trace plot of likelihood for case 4 (\cref{eq:synth:tau_x:case4}) with Gumbel copula. The top left denotes analysis with InvGamma(1,1), top right denotes analysis with InvGamma(2,2), bottom left denotes analysis with LogNormal(0,1) and bottom right denotes analysis with LogNormal(0,5).}
	\label{fig:case4:gumbel:like}
\end{figure}



\bibliographystyle{plainnat}
\bibliography{example}

\end{document}
